\documentclass{amsart}

\usepackage[margin=.75in]{geometry}
\usepackage{enumerate, amssymb, amsmath, amsthm, txfonts, wasysym, tikz, graphicx}
\usetikzlibrary{trees}
\graphicspath{{}}

\begin{document}
\title{Math 352 - Project}
\author{Sierra Battan, Jack Brockway, \(\&\) Randy Chen}
\maketitle

\textbf{Program Guide:}
\begin{enumerate}
	\item[\(\bullet\)] \(NCIS.java\) is our main file that reads user input, moves the points into our data structures, calls both our \(tridAlgo\) function (our tridiagonal system solving subroutine) and our \(ncisAlgo\) functions, and prints our findings.
	\item[\(\bullet\)] Note: these are the assumptions we have made for our program:
	\begin{enumerate}
		\item[\(*\)] All inputs are of numeric form (i.e., ''cat dog'' is an invalid point).
		\item[\(*\)] All inputs are entered in ascending order.
	\end{enumerate}
	\item[\(\bullet\)] \textbf{Usage Instructions:}
	\begin{enumerate}
		\item[\(*\)] Launch a terminal
		\item[\(*\)] Navigate to the folder: \(BBC352Project\)
		\item[\(*\)] Compile our program: \(javac NCIS.java\)
		\item[\(*\)] Launch our program: \(java NCIS\)
		\item[\(*\)] Input all points on new lines: \(x \ y\)
		\item[\(*\)] When done entering points, write: \(done\)
		\item[\(*\)] Our program will then output the resulting, simplified, natural cubic interpolation spline. \smiley
	\end{enumerate}
\end{enumerate}

\smallskip
\textbf{Natural Cubic Interpolation Splines:}
\begin{enumerate} 
	\item[] Given a table of data points, this algorithm aims to define an interpolating cubic spline function whose knots coincide with the ascending \(x\) values of the data points. A piecewise function must meet several conditions to qualify as a natural cubic interpolating spline. First, the function \(S(x)\) consists of \(n-1\) pieces (each a polynomial of degree \(\leq 3\)), defined from \([t_i, t_{i+1}]\). In addition, the polynomial satisfies the interpolation conditions, \(S(t_i) = y_i\). Furthermore, the continuity conditions are imposed at interior knots, such that \(S^k(t_i) = S^k(t_{i+1})\) at each border point. Finally, since the function is natural, we make the choice that \(S'' (t_0) = S'' (t_n) = 0\) for our two extra freedoms. After setting these conditions, we then run Gaussian Elimination on a tridiagonal system to solve for a myriad of variables to result in the polynomials: \(S_i(x) = \frac{z_{i+1}}{6h_i}(x-t_i)^3 + \frac{z_{i}}{6h_i}(t_{i+1}-x)^3 + C_i(x-t_i) + D_i(t_{i+1} -x)\), which, for simplification purposes for our computer algorithm, we choose to rewrite in the form: \(S_i(x) = D_i + C_i(x - t_i) + B_i(x - t_i)^2 + A_i(x - t_i)^3\). The resulting piecewise \(C^2\) function, \(S(x)\), is then called a natural cubic interpolation spline.
\end{enumerate}
		
\textbf{Test Cases:}
\begin{enumerate}

	\item[\(\bullet\)] Test 1:
	\begin{enumerate}
		\item[] Input: \((1, 2), (2, 3), (3, 5)\)
		\item[] Result: \(S(x) = \)
		\begin{enumerate}
			\item[] \(S_0(x) = 2.0 + 0.75(x - 1.0) + 0.0(x - 1.0)^2 + 0.25(x - 1.0)^3 \ \ \ \ (0.0 \leq x \leq1.0)\) 
			\item[] \(S_1(x) = 3.0 + 1.5(x - 2.0) + 0.75(x - 2.0)^2 + -0.25(x - 2.0)^3\ \ \ \ (1.0 \leq x \leq2.0)\)
		\end{enumerate}
	\end{enumerate}
		
	\item[\(\bullet\)] Test 2: (Textbook Problem \(9.2.32\))
	\begin{enumerate}
		\item[] Input: \((1, 0), (2, 1), (3, 0), (4, 1), (5, 0))\)
		\item[] Result: \(S(x) = \)
		\begin{enumerate}
			\item[] \(S_0(x) = 0.0 + 1.714\ldots(x - 1.0) + 0.0(x - 1.0)^2 + -0.714\ldots(x - 1.0)^3\ \ \ \ (0.0 \leq x \leq1.0)\)
			\item[] \(S_1(x) = 1.0 + -0.428\ldots(x - 2.0) + -2.142\ldots(x - 2.0)^2 + 1.571\ldots(x - 2.0)^3\ \ \ \ (1.0 \leq x \leq2.0)\)
			\item[] \(S_2(x) = 0.0 + 1.110\ldots E-16(x - 3.0) + 2.571\ldots(x - 3.0)^2 + -1.571\ldots(x - 3.0)^3\ \ \ \ (2.0 \leq x \leq3.0)\)
			\item[] \(S_3(x) = 1.0 + 0.428\ldots(x - 4.0) + -2.142\ldots(x - 4.0)^2 + 0.714\ldots(x - 4.0)^3\ \ \ \ (3.0 \leq x \leq4.0)\)
		\end{enumerate}
	\end{enumerate}
	
	\item[\(\bullet\)] Test 3: (Textbook Problem \(9.2.41)\)
	\begin{enumerate}
		\item[] Input: \((0, 1), (1, 2), (2, 3), (3, 4), (4, 5)\)
		\item[] Result: \(S(x) = \)
		\begin{enumerate}
			\item[] \(S_0(x) = 1.0 + 1.0(x - 0.0) + 0.0(x - 0.0)^2 + 0.0(x - 0.0)^3\ \ \ \ (0.0 \leq x \leq1.0)\)
			\item[] \(S_1(x) = 2.0 + 1.0(x - 1.0) + 0.0(x - 1.0)^2 + 0.0(x - 1.0)^3\ \ \ \ (1.0 \leq x \leq2.0)\)
			\item[] \(S_2(x) = 3.0 + 1.0(x - 2.0) + 0.0(x - 2.0)^2 + 0.0(x - 2.0)^3\ \ \ \ (2.0 \leq x \leq3.0)\)
			\item[] \(S_3(x) = 4.0 + 1.0(x - 3.0) + 0.0(x - 3.0)^2 + 0.0(x - 3.0)^3\ \ \ \ (3.0 \leq x \leq4.0)\)
		\end{enumerate}
	\end{enumerate}
\end{enumerate}		
		
\end{document}