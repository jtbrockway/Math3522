\documentclass{amsart}

\usepackage[margin=1in]{geometry}
\usepackage{enumerate, amssymb, amsmath, amsthm, txfonts, wasysym, tikz, graphicx}
\usetikzlibrary{trees}
\graphicspath{{}}

\begin{document}
\title{Math 352 - Project}
\author{Sierra Battan, Jack Brockway, \(\&\) Randy Chen}
\maketitle

\begin{enumerate}

	\item[] \textbf{Program Guide:}
	\begin{enumerate}
		\item[\(\bullet\)] \(NCIS.java\) is our main program files that reads the user's input, moves the points into our data structures, calls our \(tridAlgo\) function (which is our subroutine that solves a tridiagonal system of equations), calls our \(ncisAlgo\) function, and prints out our findings.
		\item[\(\bullet\)] Note: these are the assumptions we have made for our program:
		\begin{enumerate}
			\item[\(*\)] All inputs are of numeric form (i.e., ''cat dog'' is an invalid point).
			\item[\(*\)] All inputs are entered in ascending order.
		\end{enumerate}
		\item[\(\bullet\)] \textbf{Usage Instructions:}
		\begin{enumerate}
			\item[\(*\)] Launch a terminal
			\item[\(*\)] Navigate to the folder: \(BBC352Project\)
			\item[\(*\)] Compile our program: \(javac NCIS.java\)
			\item[\(*\)] Launch our program: \(java NCIS\)
			\item[\(*\)] Input all points on new lines: \(x \ y\)
			\item[\(*\)] When done entering points, write: \(done\)
			\item[\(*\)] Our program will then output the resulting, simplified, natural cubic interpolation spline. \smiley
		\end{enumerate}
	\end{enumerate}
	
	\smallskip
	\item[] \textbf{Natural Cubic Interpolation Splines:}
	\begin{enumerate} 
		\item[] Natural cubic interpolation splines are blah blah blah. You can solved them by first blah blah blah, using blah blah blah. The result is a piecewise, \(C^2\) function, which can be rewritten in many different ways.  
	\end{enumerate}
		
	\item[] \textbf{Test Cases:}
	\begin{enumerate}

		\item[\(\bullet\)] Test 1:
		\begin{enumerate}
			\item[] Input: \((1, 2), (2, 3), (3, 5)\)
			\item[] Result: \(S(x) = \)
			\begin{enumerate}
				\item[] \(S_0(x) = 2.0 + 0.75(x - 1.0) + 0.0(x - 1.0)^2 + 0.25(x - 1.0)^3\)
				\item[] \(S_1(x) = 3.0 + 1.5(x - 2.0) + 0.75(x - 2.0)^2 + -0.25(x - 2.0)^3\)
			\end{enumerate}
		\end{enumerate}
			
		\item[\(\bullet\)] Test 2: (Textbook Problem \(9.2.32\))
		\begin{enumerate}
			\item[] Input: \((1, 0), (2, 1), (3, 0), (4, 1), (5, 0))\)
			\item[] Result: \(S(x) = \)
			\begin{enumerate}
				\item[] \(S_0(x) = 0.0 + 1.7142857\ldots(x - 1.0) + 0.0(x - 1.0)^2 + -0.7142857\ldots(x - 1.0)^3\)
				\item[] \(S_1(x) = 1.0 + -0.4285714\ldots(x - 2.0) + -2.1428571\ldots(x - 2.0)^2 + 1.5714285\ldots(x - 2.0)^3\)
				\item[] \(S_2(x) = 0.0 + 1.1102230\ldots E-16(x - 3.0) + 2.5714285\ldots(x - 3.0)^2 + -1.5714285\ldots(x - 3.0)^3\)
				\item[] \(S_3(x) = 1.0 + 0.4285714\ldots(x - 4.0) + -2.1428571\ldots(x - 4.0)^2 + 0.7142857\ldots(x - 4.0)^3\)
			\end{enumerate}
		\end{enumerate}
		
		\item[\(\bullet\)] Test 3: (Textbook Problem \(9.2.41)\)
		\begin{enumerate}
			\item[] Input: \((0, 1), (1, 2), (2, 3), (3, 4), (4, 5)\)
			\item[] Result: \(S(x) = \)
			\begin{enumerate}
				\item[] \(S_0(x) = 1.0 + 1.0(x - 0.0) + 0.0(x - 0.0)^2 + 0.0(x - 0.0)^3\)
				\item[] \(S_1(x) = 2.0 + 1.0(x - 1.0) + 0.0(x - 1.0)^2 + 0.0(x - 1.0)^3\)
				\item[] \(S_2(x) = 3.0 + 1.0(x - 2.0) + 0.0(x - 2.0)^2 + 0.0(x - 2.0)^3\)
				\item[] \(S_3(x) = 4.0 + 1.0(x - 3.0) + 0.0(x - 3.0)^2 + 0.0(x - 3.0)^3\)
			\end{enumerate}
		\end{enumerate}
	\end{enumerate}		
		
\end{enumerate}		
\end{document}